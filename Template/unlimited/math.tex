\section{int64相乘取模\ \small(Durandal)}
    \myminted{cpp}{math/int64_multiply_mod.cpp}

\section{ex-Euclid\ \small(gy)}
    \myminted{cpp}{math/extend_gcd.cpp}

\section{中国剩余定理\ \small(Durandal)}
    \subsection*{中国剩余定理}
        若$ m_1, m_2, \dots, m_n $两两互素,则方程组$ x \equiv a_i \pmod{m_i} $有解,且解可以由以下方法构造:
        \\$ M = \prod\limits_{i = 1}^{n} m_i $
        \\$ M_i = \dfrac{M}{m_i} $
        \\$ t_i = \dfrac{1}{M_i} \bmod m_i $
        \\$ ans \equiv \sum\limits_{i = 1}^{n} a_i t_i M_i \pmod M $
    \subsection*{ex-Euclid解二元中国剩余定理}
        返回是否可行,余数和模数结果为$ r_1 $, $ m_1 $
        \myminted{cpp}{math/chinese_remainder_theorem.cpp}

\section{线性同余不等式\ \small(Durandal)}
    必须满足$ 0 \leq d < m $, $ 0 \leq l \leq r < m $,返回$ \min\lbrace x \geq 0 \mid l \leq x \cdot d \bmod m \leq r \rbrace $,无解返回$ -1 $ 
    \myminted{cpp}{math/linear_congruence_inequality.cpp}

\section{平方剩余\ \small(Nightfall)}
    $ x^2 \equiv a \pmod p, 0 \leq a < p $
    \\返回是否存在解
    \\$ p $必须是质数, 若是多个单次质数的乘积可以分别求解再用CRT合并
    \\复杂度为$ O(\log n) $
    \myminted{cpp}{math/square_remainder.cpp}

\section{组合数\ \small(Nightfall)}
    \myminted{cpp}{math/ex_lucas.cpp}

\section{高斯消元\ \small(ct)}
    \myminted{cpp}{math/gauss_elimination.cpp}

\section{Miller\ Rabin\ \&\ Pollard\ Rho\ \small(gy)}
    \begin{tabular}{l r}
        \hline
        Test Set & First Wrong Answer\\\hline
        $ 2, 3, 5, 7 $ & (INT32\_MAX)\\\hline
        $ 2, 7, 61 $ & $ 4,759,123,141 $\\\hline
        $ 2, 3, 5, 7, 11, 13, 17, 19, 23, 29, 31, 37 $ & (INT64\_MAX)\\\hline
    \end{tabular}
    \myminted{cpp}{math/miller_rabin_and_pollard_rho.cpp}

\section{$ O(m ^ 2 \log n) $线性递推\ \small(lhy)}
    \myminted{cpp}{math/linear_rec.cpp}

\section{线性基\ \small(ct)}
    \myminted{cpp}{math/linear_base.cpp}

\section{多项式\ \small(lhy,ct,gy)}
    \subsection*{FFT\ \small(ct)}
        \myminted{cpp}{math/fft.cpp}
    \subsection*{MTT\ \small(gy)}
        \myminted{cpp}{math/mtt.cpp}
    \subsection*{NTT\ \small(gy)}
        \begin{tabular}{l c r}
            \hline
            Prime & $ G $ & $ 2^k $\\\hline
            $ 167772161 $ & $ 3 $ & $ 33554432 $\\\hline
            $ 469762049 $ & $ 3 $ & $ 67108864 $\\\hline
            $ 998244353 $ & $ 3 $ & $ 8388608 $\\\hline
            $ 1004535809 $ & $ 3 $ & $ 2097152 $\\\hline
        \end{tabular}
        \myminted{cpp}{math/ntt.cpp}
    \subsection*{FWT\ \small(lhy)}
        \myminted{cpp}{math/fwt.cpp}
    \subsection*{多项式操作\ \small(gy)}
        \begin{itemize}[wide=0pt]
            \item 求逆:
            \\$ A(x) B(x) \equiv 1 \pmod{x^t} \to A(x) (2B(x) - A(x) B^2(x)) \equiv 1 \pmod{x^{2t}} $
            \item 平方根:
            \\$ A^2(x) \equiv B(x) \pmod{x^t} \to (\dfrac{B(x) + A^2(x)}{2A(x)})^2 \equiv B(x) \pmod{x^{2t}} $
            \item $ \ln $(常数项为1):
            \\$ A(x) = \ln B(x) \to A'(x) = \dfrac{B'(x)}{B(x)} $
            \item $ \exp $(常数项为0):
            \\$ B(x) \equiv e^{A(x)} \pmod{x^t} \to B(x) (1 - \ln B(x) + A(x)) \equiv e^{A(x)} \pmod{x^{2t}} $
        \end{itemize}
        \myminted{cpp}{math/poly_operation.cpp}

\section{筛\ \small(ct,cxy,Nightfall)}

    \subsection*{杜教筛\ \small(ct)}
        \noindent Dirichlet卷积:$ (f \ast g) (n) = \sum\limits_{d \mid n}^{} f(d) g(\dfrac{n}{d}) $
        \\对于积性函数$ f(n) $,求其前缀和$ S(n) = \sum\limits_{i = 1}^{n} f(i) $
        \\寻找一个恰当的积性函数$ g(n) $,使得$ g(n) $和$ (f \ast g) (n) $的前缀和都容易计算
        \\则$ g(1) S(n) = \sum\limits_{i = 1}^{n} (f \ast g) (i) - \sum\limits_{i = 2}{n} g(i) S(\lfloor \dfrac{n}{i} \rfloor) $
        \\$ \mu (n) $和$ \phi (n) $取$ g(n) = 1 $
        \\两种常见形式:
        \begin{itemize}[wide=0pt]
            \item $ S(n) = \sum\limits_{i = 1}^{n} (f \cdot g) (i) $且$ g(i) $为完全积性函数
                \\$ S(n) = \sum\limits_{i = 1}^{n} ((f \ast 1) \cdot g) (i) - \sum\limits_{i = 2}^{n} S(\lfloor \dfrac{n}{i} \rfloor) g(i) $
            \item $ S(n) = \sum\limits_{i = 1}^{n} (f \ast g) (i) $
                \\$ S(n) = \sum\limits_{i = 1}^{n} g (i) \sum\limits_{ij \leq n}^{} (f \ast 1) (j) - \sum\limits_{i = 2}^{n} S(\lfloor \dfrac{n}{i} \rfloor) $
        \end{itemize}
        \myminted{cpp}{math/du_jiao_sieve.cpp}

    \subsection*{Extended\ Eratosthenes\ Sieve\ \small(Nightfall)}
        一般积性函数的前缀和,要求:$ f(p) $为多项式
        \myminted{cpp}{math/ex_eratosthenes_sieve.cpp}

    \subsection*{Min25筛\ \small(cxy)}
        \subsubsection*{Min25筛}
            $O(\dfrac{n^\dfrac{3}{4}}{\log n})$求$\sum\limits_{i = 1}^n f(i)$,其中$f(i)$是积性函数且$f(p^e)$是关于$p$的低阶多项式
            \\$g(n, j) = \sum\limits_{i = 1}^n [i~\text{is a prime or}~\min_p(i) > p_j] i^k$
            \\$g(n, j) = g(n, j - 1) - p_j^k\left( g\left( \dfrac{n}{p_j}, j - 1 \right) - g\left( p_{j - 1}, j - 1 \right) \right)$
            \\$g(n) = g(n, x)$,其中$p_x$是最后一个$\le \sqrt n$的素数
            \\$sp_n = \sum\limits_{i = 1}^n p_i^k$
            \\$S(n, x) = g(n) - sp_x + \sum\limits_{p_k^e \le n \wedge k > x} f(p_k^e) \left( S\left( \dfrac{n}{p_k^e}, k \right) + [e \ne 1] \right) $
            \\$\sum\limits_{i = 1}^n = S(n, 0)$
        \subsubsection*{实现$f(p^k) = p^k(p^k - 1)$}
            \myminted{cpp}{math/min25.cpp}

    \subsection*{洲阁筛\ \small(ct)}
        \subsubsection*{实现区间$[a, b]$里有多少个数满足:含有至少一个$>p$的素因数}
            \myminted{cpp}{math/zhou_sieve_1.cpp}
        \subsubsection*{实现$\sum\limits_{i=1}^n \sum\limits_{j=1}^n \operatorname{sgcd}(i, j)^k$, 其中$\operatorname{sgcd}(i, j) = \begin{cases} 0 & \gcd(i, j) = 1 \\ \text{The second-largest common divisor of}~i~\text{and}~j & \text{otherwise} \end{cases}$}
            \myminted{cpp}{math/zhou_sieve_2.cpp}

\section{BSGS\ \small(ct,Durandal)}
    \subsection*{BSGS\ \small(ct)}
        $ p $是素数,返回$ \min\lbrace x \geq 0 \mid y^x \equiv z \pmod p \rbrace $
        \myminted{cpp}{math/bsgs.cpp}
    \subsection*{ex-BSGS\ \small(Durandal)}
        必须满足$ 0 \leq a < p $, $ 0 \leq b < p $,返回$ \min\lbrace x \geq 0 \mid a^x \equiv b \pmod p\rbrace $
        \myminted{cpp}{math/ex_bsgs.cpp}

\section{直线下整点个数\ \small(gy)}
    必须满足$ a \geq 0 $, $ b \geq 0 $, $ m > 0 $,返回$ \sum\limits_{i=0}^{n-1} \dfrac{a + bi}{m} $
    \myminted{cpp}{math/points_below_line.cpp}

\section{Pell\ equation\ \small(gy)}
    $ x^2 - n y^2 = 1 $有解当且仅当$ n $不为完全平方数
    \\求其特解$ (x_0, y_0) $
    \\其通解为$ (x_{k + 1}, y_{k + 1}) = (x_0 x_k + n y_0 y_k, x_0 y_k + y_0 x_k ) $
    \myminted{cpp}{math/pell.cpp}

\section{单纯形\ \small(gy)}
    返回$ x_{m \times 1} $使得$ \max \lbrace c_{1 \times m} \cdot x_{m \times 1} \mid x_{m \times 1} \geq 0_{m \times 1}, A_{n \times m} \cdot x_{m \times 1} \leq b_{n \times 1} \rbrace $
    \myminted{cpp}{math/simplex.cpp}

\section{数学知识\ \small(gy)}
    \subsection*{Pick theorem}
    顶点为整点的简单多边形,其面积$ A $,内部格点数$ i $,边上格点数$ b $满足:
    \\$ A = i + \dfrac{b}{2} - 1 $

\subsection*{欧拉示性数}
    \begin{itemize}[wide=0pt]
        \item 三维凸包的顶点个数$ V $,边数$ E $,面数$ F $满足:
        \\$ V - E + F = 2 $
        \item 平面图的顶点个数$ V $,边数$ E $,平面被划分的区域数$ F $,组成图形的连通部分的数目$ C $满足:
        \\$ V - E + F = C + 1 $
    \end{itemize}

\subsection*{闵可夫斯基和}
    \noindent $A, B$的闵可夫斯基和定义为$C = \{ a + b \mid a \in A, b \in B \}$\\
    闵可夫斯基和的边是由两凸包构成的,把两凸包的边极角排序后顺次连接起来就是闵可夫斯基和。需要注意的是可能会有三点共线的情况,因此需要重新求一次凸包

\subsection*{最小乘积生成树}
    \noindent 找到$\sum\limits_i x_i \cdot \sum\limits_i y_i$最小的的生成树\\
    首先找出两个在凸包上的点$A(\min_x, y), B(x, \min_y)$,在直线$AB$下方找一个在凸包上且$x \cdot y$最小的点。可以每次找距离直线$AB$最远的点,记每条边的$y_i = y_i \cdot (x_B, x_A), x_i = x_i \cdot (y_A, y_B)$,以$x_i + y_i$为关键字排序做Kruskal。递归处理直到叉积$\ge 0$

\subsection*{几何公式}
    \begin{itemize}[nosep, wide=0pt]
        \item \textbf{三角形}
            \\半周长$ p = \dfrac{a + b + c}{2} $
            \\面积$ S = \dfrac{1}{2} a H_a = \dfrac{1}{2} a b \cdot \sin C = \sqrt{p(p - a)(p - b)(p - c)} = p r = \dfrac{a b c}{4R} $
            \\中线长$ M_a = \dfrac{1}{2} \sqrt{2(b^2 + c^2) - a^2} = \dfrac{1}{2} \sqrt{b^2 + c^2 + 2 b c \cdot \cos A} $
            \\角平分线长$ T_a = \dfrac{\sqrt{bc((b + c)^2 - a^2)}}{b + c} = \dfrac{2 b c}{b + c} \cos \dfrac{A}{2} $
            \\高$ H_a = b \sin C = \sqrt{b^2 - (\dfrac{a^2 + b^2 - c^2}{2 a})^2} $
            \\内切圆半径$ r = \dfrac{S}{p} = 4 R \sin \dfrac{A}{2} \sin \dfrac{B}{2} \sin \dfrac{C}{2} = \sqrt{\dfrac{(p - a)(p - b)(p - c)}{p}} = p \tan \dfrac{A}{2} \tan \dfrac{B}{2} \tan \dfrac{C}{2} $
            \\外接圆半径$ R = \dfrac{a b c}{4 S} = \dfrac{a}{2 \sin A} $
            \\旁切圆半径$ r_A = \dfrac{2 S}{- a + b + c} $
            \\重心$ (\dfrac{x_1 + x_2 + x_3}{3}, \dfrac{y_1 + y_2 + y_3}{3}) $
            \\外心$ (\dfrac{\left|\begin{array}{cccc}
                x_1^2 + y_1^2 & y_1 & 1\\
                x_2^2 + y_2^2 & y_2 & 1\\
                x_3^2 + y_3^2 & y_3 & 1
                \end{array}\right|}{2 \left|\begin{array}{cccc}
                x_1 & y_1 & 1\\
                x_2 & y_2 & 1\\
                x_3 & y_3 & 1
                \end{array}\right|}, \dfrac{\left|\begin{array}{cccc}
                x_1 & x_1^2 + y_1^2 & 1\\
                x_2 & x_2^2 + y_2^2 & 1\\
                x_3 & x_3^2 + y_3^2 & 1
                \end{array}\right|}{2\left|\begin{array}{cccc}
                x_1 & y_1 & 1\\
                x_2 & y_2 & 1\\
                x_3 & y_3 & 1
                \end{array}\right|}) $
            \\内心$ (\dfrac{a x_1 + b x_2 + c x_3}{a + b + c}, \dfrac{a y_1 + b y_2 + c y_3}{a + b + c}) $
            \\垂心$ (\dfrac{\left|\begin{array}{cccc}
                x_2 x_3 + y_2 y_3 & 1 & y_1\\
                x_3 x_1 + y_3 y_1 & 1 & y_2\\
                x_1 x_2 + y_1 y_2 & 1 & y_3
                \end{array}\right|}{2 \left|\begin{array}{cccc}
                x_1 & y_1 & 1\\
                x_2 & y_2 & 1\\
                x_3 & y_3 & 1
                \end{array}\right|}, \dfrac{\left|\begin{array}{cccc}
                x_2 x_3 + y_2 y_3 & x_1 & 1\\
                x_3 x_1 + y_3 y_1 & x_2 & 1\\
                x_1 x_2 + y_1 y_2 & x_3 & 1
                \end{array}\right|}{2\left|\begin{array}{cccc}
                x_1 & y_1 & 1\\
                x_2 & y_2 & 1\\
                x_3 & y_3 & 1
                \end{array}\right|}) $
            \\旁心$ (\dfrac{- a x_1 + b x_2 + c x_3}{- a + b + c}, \dfrac{- a y_1 + b y_2 + c y_3}{- a + b + c}) $
            \\Trillinear coordinates: $ \dfrac{ax}{ax + by + cz} A + \dfrac{by}{ax + by + cz} B + \dfrac{cz}{ax + by + cz} C $
            \\$ x $, $ y $, $ z $分别代表点$ P $到边的距离
            \\Fermat point: $ x : y : z = \csc(A + \dfrac{\pi}{3}) : \csc(B + \dfrac{\pi}{3}) : \csc(C + \dfrac{\pi}{3}) $
        \item \textbf{圆}
            \\弧长$ l = r A $
            \\弦长$ a = 2 \sqrt{2 h r - h^2} = 2 r \cdot \sin \dfrac{A}{2} $
            \\弓形高$ h = r - \sqrt{r^2 - \dfrac{a^2}{4}} = r (1 - \cos \dfrac{A}{2}) $
            \\扇形面积 $ S_1 = \dfrac{1}{2} l r = \dfrac{1}{2} A r^2 $
            \\弓形面积 $ S_2 = \dfrac{1}{2} r^2 (A - \sin A) $
        \item \textbf{Circles of Apollonius}
            \\已知三个两两相切的圆,半径为$ r_1, r_2, r_3 $
            \\与它们外切的圆半径为$ \left| \dfrac{r_1 r_2 r_3}{r_1 r_2 + r_2 r_3 + r_3 r_1 - 2 \sqrt{r_1 r_2 r_3 (r_1 + r_2 + r_3)}} \right| $
            \\与它们内切的圆半径为$ \dfrac{r_1 r_2 r_3}{r_1 r_2 + r_2 r_3 + r_3 r_1 + 2 \sqrt{r_1 r_2 r_3 (r_1 + r_2 + r_3)}} $
        \item \textbf{棱台}
            \\体积$ V = \dfrac{1}{3} h (A_1 + A_2 + \sqrt{A_1 A_2}) $
            \\正棱台侧面积$ S = \dfrac{1}{2} (p_1 + p_2) l $,$ l $为侧高
        \item \textbf{球}
            \\体积$ V = \dfrac{4}{3} \pi r^3 $
            \\表面积$ S = 4 \pi r^2 $
        \item \textbf{球台}
            \\侧面积$ S = 2 \pi r h $
            \\体积$ V = \dfrac{1}{6} \pi h (3(r_1^2 + r_2^2) + h_h) $
        \item \textbf{球扇形}
            \\球面面积$ S = 2 \pi r h $
            \\体积$ V = \dfrac{2}{3} \pi r^2 h = \dfrac{2}{3} \pi r^3 h (1 - \cos \varphi) $
        \item \textbf{球面三角形}
            \\考虑单位球上的球面三角形,$ a, b, c $表示三边长(弧所对球心角),$ A, B, C $表示三角大小(切线夹角)
            \\余弦定理$ \cos a = \cos b \cdot \cos c + \sin a \cdot \sin b \cdot \cos A $
            \\正弦定理$ \dfrac{\sin A}{\sin a} = \dfrac{\sin B}{\sin b} = \dfrac{\sin C}{\sin c} $
            \\球面面积$ S = A + B + C - \pi $
        \item \textbf{四面体}
            \\体积$ V = \dfrac{1}{6} \left| \overrightarrow{AB} \cdot (\overrightarrow{AC} \times \overrightarrow{AD}) \right| $
    \end{itemize}

