\subsection*{Pick theorem}
    顶点为整点的简单多边形,其面积$ A $,内部格点数$ i $,边上格点数$ b $满足:
    \\$ A = i + \dfrac{b}{2} - 1 $

\subsection*{欧拉示性数}
    \begin{itemize}[wide=0pt]
        \item 三维凸包的顶点个数$ V $,边数$ E $,面数$ F $满足:
        \\$ V - E + F = 2 $
        \item 平面图的顶点个数$ V $,边数$ E $,平面被划分的区域数$ F $,组成图形的连通部分的数目$ C $满足:
        \\$ V - E + F = C + 1 $
    \end{itemize}

\subsection*{闵可夫斯基和}
    \noindent $A, B$的闵可夫斯基和定义为$C = \{ a + b \mid a \in A, b \in B \}$\\
    闵可夫斯基和的边是由两凸包构成的,把两凸包的边极角排序后顺次连接起来就是闵可夫斯基和。需要注意的是可能会有三点共线的情况,因此需要重新求一次凸包

\subsection*{最小乘积生成树}
    \noindent 找到$\sum\limits_i x_i \cdot \sum\limits_i y_i$最小的的生成树\\
    首先找出两个在凸包上的点$A(\min_x, y), B(x, \min_y)$,在直线$AB$下方找一个在凸包上且$x \cdot y$最小的点。可以每次找距离直线$AB$最远的点,记每条边的$y_i = y_i \cdot (x_B, x_A), x_i = x_i \cdot (y_A, y_B)$,以$x_i + y_i$为关键字排序做Kruskal。递归处理直到叉积$\ge 0$

\subsection*{几何公式}
    \begin{itemize}[nosep, wide=0pt]
        \item \textbf{三角形}
            \\半周长$ p = \dfrac{a + b + c}{2} $
            \\面积$ S = \dfrac{1}{2} a H_a = \dfrac{1}{2} a b \cdot \sin C = \sqrt{p(p - a)(p - b)(p - c)} = p r = \dfrac{a b c}{4R} $
            \\中线长$ M_a = \dfrac{1}{2} \sqrt{2(b^2 + c^2) - a^2} = \dfrac{1}{2} \sqrt{b^2 + c^2 + 2 b c \cdot \cos A} $
            \\角平分线长$ T_a = \dfrac{\sqrt{bc((b + c)^2 - a^2)}}{b + c} = \dfrac{2 b c}{b + c} \cos \dfrac{A}{2} $
            \\高$ H_a = b \sin C = \sqrt{b^2 - (\dfrac{a^2 + b^2 - c^2}{2 a})^2} $
            \\内切圆半径$ r = \dfrac{S}{p} = 4 R \sin \dfrac{A}{2} \sin \dfrac{B}{2} \sin \dfrac{C}{2} = \sqrt{\dfrac{(p - a)(p - b)(p - c)}{p}} = p \tan \dfrac{A}{2} \tan \dfrac{B}{2} \tan \dfrac{C}{2} $
            \\外接圆半径$ R = \dfrac{a b c}{4 S} = \dfrac{a}{2 \sin A} $
            \\旁切圆半径$ r_A = \dfrac{2 S}{- a + b + c} $
            \\重心$ (\dfrac{x_1 + x_2 + x_3}{3}, \dfrac{y_1 + y_2 + y_3}{3}) $
            \\外心$ (\dfrac{\left|\begin{array}{cccc}
                x_1^2 + y_1^2 & y_1 & 1\\
                x_2^2 + y_2^2 & y_2 & 1\\
                x_3^2 + y_3^2 & y_3 & 1
                \end{array}\right|}{2 \left|\begin{array}{cccc}
                x_1 & y_1 & 1\\
                x_2 & y_2 & 1\\
                x_3 & y_3 & 1
                \end{array}\right|}, \dfrac{\left|\begin{array}{cccc}
                x_1 & x_1^2 + y_1^2 & 1\\
                x_2 & x_2^2 + y_2^2 & 1\\
                x_3 & x_3^2 + y_3^2 & 1
                \end{array}\right|}{2\left|\begin{array}{cccc}
                x_1 & y_1 & 1\\
                x_2 & y_2 & 1\\
                x_3 & y_3 & 1
                \end{array}\right|}) $
            \\内心$ (\dfrac{a x_1 + b x_2 + c x_3}{a + b + c}, \dfrac{a y_1 + b y_2 + c y_3}{a + b + c}) $
            \\垂心$ (\dfrac{\left|\begin{array}{cccc}
                x_2 x_3 + y_2 y_3 & 1 & y_1\\
                x_3 x_1 + y_3 y_1 & 1 & y_2\\
                x_1 x_2 + y_1 y_2 & 1 & y_3
                \end{array}\right|}{2 \left|\begin{array}{cccc}
                x_1 & y_1 & 1\\
                x_2 & y_2 & 1\\
                x_3 & y_3 & 1
                \end{array}\right|}, \dfrac{\left|\begin{array}{cccc}
                x_2 x_3 + y_2 y_3 & x_1 & 1\\
                x_3 x_1 + y_3 y_1 & x_2 & 1\\
                x_1 x_2 + y_1 y_2 & x_3 & 1
                \end{array}\right|}{2\left|\begin{array}{cccc}
                x_1 & y_1 & 1\\
                x_2 & y_2 & 1\\
                x_3 & y_3 & 1
                \end{array}\right|}) $
            \\旁心$ (\dfrac{- a x_1 + b x_2 + c x_3}{- a + b + c}, \dfrac{- a y_1 + b y_2 + c y_3}{- a + b + c}) $
            \\Trillinear coordinates: $ \dfrac{ax}{ax + by + cz} A + \dfrac{by}{ax + by + cz} B + \dfrac{cz}{ax + by + cz} C $
            \\$ x $, $ y $, $ z $分别代表点$ P $到边的距离
            \\Fermat point: $ x : y : z = \csc(A + \dfrac{\pi}{3}) : \csc(B + \dfrac{\pi}{3}) : \csc(C + \dfrac{\pi}{3}) $
        \item \textbf{圆}
            \\弧长$ l = r A $
            \\弦长$ a = 2 \sqrt{2 h r - h^2} = 2 r \cdot \sin \dfrac{A}{2} $
            \\弓形高$ h = r - \sqrt{r^2 - \dfrac{a^2}{4}} = r (1 - \cos \dfrac{A}{2}) $
            \\扇形面积 $ S_1 = \dfrac{1}{2} l r = \dfrac{1}{2} A r^2 $
            \\弓形面积 $ S_2 = \dfrac{1}{2} r^2 (A - \sin A) $
        \item \textbf{Circles of Apollonius}
            \\已知三个两两相切的圆,半径为$ r_1, r_2, r_3 $
            \\与它们外切的圆半径为$ \left| \dfrac{r_1 r_2 r_3}{r_1 r_2 + r_2 r_3 + r_3 r_1 - 2 \sqrt{r_1 r_2 r_3 (r_1 + r_2 + r_3)}} \right| $
            \\与它们内切的圆半径为$ \dfrac{r_1 r_2 r_3}{r_1 r_2 + r_2 r_3 + r_3 r_1 + 2 \sqrt{r_1 r_2 r_3 (r_1 + r_2 + r_3)}} $
        \item \textbf{棱台}
            \\体积$ V = \dfrac{1}{3} h (A_1 + A_2 + \sqrt{A_1 A_2}) $
            \\正棱台侧面积$ S = \dfrac{1}{2} (p_1 + p_2) l $,$ l $为侧高
        \item \textbf{球}
            \\体积$ V = \dfrac{4}{3} \pi r^3 $
            \\表面积$ S = 4 \pi r^2 $
        \item \textbf{球台}
            \\侧面积$ S = 2 \pi r h $
            \\体积$ V = \dfrac{1}{6} \pi h (3(r_1^2 + r_2^2) + h_h) $
        \item \textbf{球扇形}
            \\球面面积$ S = 2 \pi r h $
            \\体积$ V = \dfrac{2}{3} \pi r^2 h = \dfrac{2}{3} \pi r^3 h (1 - \cos \varphi) $
        \item \textbf{球面三角形}
            \\考虑单位球上的球面三角形,$ a, b, c $表示三边长(弧所对球心角),$ A, B, C $表示三角大小(切线夹角)
            \\余弦定理$ \cos a = \cos b \cdot \cos c + \sin a \cdot \sin b \cdot \cos A $
            \\正弦定理$ \dfrac{\sin A}{\sin a} = \dfrac{\sin B}{\sin b} = \dfrac{\sin C}{\sin c} $
            \\球面面积$ S = A + B + C - \pi $
        \item \textbf{四面体}
            \\体积$ V = \dfrac{1}{6} \left| \overrightarrow{AB} \cdot (\overrightarrow{AC} \times \overrightarrow{AD}) \right| $
    \end{itemize}
