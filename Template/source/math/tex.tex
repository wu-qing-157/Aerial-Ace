\newcommand{\eularian}{\genfrac\langle\rangle{0pt}{2}}
%\newcommand{\Eularian}[2]{\left\langle\!\eularian{#1}{#2}\!\right\rangle}
\newcommand{\Eularian}{\genfrac{\langle\!\langle}{\rangle\!\rangle}{0pt}{2}}

\subsection*{扩展欧拉定理}
    $ a^c \equiv \begin{cases} a^c & c < \phi(m) \\ a^{c \bmod \phi(m) + \phi(m)} & c \ge \phi(m) \end{cases} $

\subsection*{类欧几里得算法}
    $ f(a, b, c, n) = \sum\limits_{i=0}^n \lfloor \frac{ai + b}{c} \rfloor, g(a, b, c, n) = \sum\limits_{i=0}^n i \lfloor \frac{ai + b}{c} \rfloor, h(a, b, c, n) = \sum\limits_{i=0}^n \lfloor \frac{ai + b}{c} \rfloor ^2, m = \lfloor \frac{an + b}{c} \rfloor $
    \\$a \ge c$ or $b \ge c$:
    \\$f(a, b, c, n) = f(a \bmod c, b \bmod c, c, n) + \frac{1}{2} n (n + 1) \lfloor \frac{a}{c} \rfloor + (n + 1) \lfloor \frac{b}{c} \rfloor $
    \\$g(a, b, c, n) = g(a \bmod c, b \bmod c, c, n) + \frac{1}{6} n (n + 1) (2n + 1) \lfloor \frac{a}{c} \rfloor + \frac{1}{2} n (n + 1) \lfloor \frac{b}{b} \rfloor $
    \\$h(a, b, c, n) = 2 \lfloor \frac{b}{c} \rfloor f(a \bmod c, b \bmod c, c, n) + 2 \lfloor \frac{a}{c} \rfloor g(a \bmod c, b \bmod c, c, n) + h(a \bmod c, b \bmod c, c, n) + \frac{1}{6} n (n + 1) (2n + 1) \lfloor \frac{a}{c} \rfloor^2 + (n + 1) \lfloor \frac{b}{c} \rfloor^2 + n (n + 1) \lfloor \frac{a}{c} \rfloor \lfloor \frac{b}{c} \rfloor $
    \\$ a < c $ and $ b < c $:
    \\$ f(a, b, c, n) = nm - f(c, c - b - 1, a, m - 1) $
    \\$ g(a, b, c, n) = \frac{1}{2} (nm(n + 1) - f(c, c - b - 1, a, m - 1) - h(c, c - b - 1, a, m - 1)) $
    \\$ h(a, b, c, n) = nm(m + 1) - f(a, b, c, n) - 2f(c, c - b - 1, a, m - 1) - 2g(c, c - b - 1, a, m - 1) $

\subsection*{原根}
    当$ \gcd(a, m) = 1 $时,使$ a^x \equiv 1 \pmod m $成立的最小正整数$ x $称为$ a $对于模$ m $的阶,计为$ \text{ord}_m(a) $。
    \\阶的性质:$ a^n \equiv 1 \pmod m $的充要条件是$ \text{ord}_m(a) \mid n $,可推出$ \text{ord}_m(a) \mid \psi(m) $。
    \\当$ \text{ord}_m(g) = \psi(m) $时,则称$ g $是模$ n $的一个原根,$ g^0, g^1, \dots, g^{\psi(m) - 1} $覆盖了$ m $以内所有与$ m $互素的数。
    \\原根存在的充要条件:$ m = 2, 4, p^k, 2 p^k $,其中$ p $为奇素数,$ k \in \mathbb{N}^\ast $

\subsection*{求和公式}
    \begin{itemize}[wide=0pt]
        \item $ \sum\limits_{k=1}^{n} (2k - 1)^2 = \frac{1}{3} n(4n^2 - 1) $
        \item $ \sum\limits_{k=1}^{n} k^3 = \frac{1}{4} n^2(n + 1)^2 $
        \item $ \sum\limits_{k=1}^{n} (2k - 1)^3 = n^2(2n^2 - 1) $
        \item $ \sum\limits_{k=1}^{n} k^4 = \frac{1}{30} n(n + 1) (2n + 1) (3n^2 + 3m - 1) $
        \item $ \sum\limits_{k=1}^{n} k^5 = \frac{1}{12} n^2(n + 1)^2(2n^2 + 2n - 1) $
        \item $ \sum\limits_{k=1}^{n} k(k + 1) = \frac{1}{3} n(n + 1)(n + 2) $
        \item $ \sum\limits_{k=1}^{n} k(k + 1)(k + 2) = \frac{1}{4} n(n + 1)(n + 2)(n + 3) $
        \item $ \sum\limits_{k=1}^{n} k(k + 1)(k + 2)(k + 3) = \frac{1}{5} n(n + 1)(n + 2)(n + 3)(n + 4) $
    \end{itemize}

\subsection*{错排公式}
    $ D_n $表示$ n $个元素错位排列的方案数
    \\$ D_1 = 0, D_2 = 1 $, $ D_n = (n - 1)(D_{n - 2} + D_{n - 1}), n \geq 3 $
    \\$ D_n = n! \cdot (1 - \frac{1}{1!} + \frac{1}{2!} - \dots + (-1)^n\frac{1}{n!}) $

\subsection*{Fibonacci sequence}
    \noindent$ F_0 = 0, F_1 = 1 $
    \\$ F_n = F_{n - 1} + F_{n - 2} $
    \\$ F_{n + 1} \cdot F_{n - 1} - F_{n}^2 = (-1)^n $
    \\$ F_{-n} = (-1)^n F_n $
    \\$ F_{n + k} = F_k \cdot F_{n + 1} + F_{k - 1} \cdot F_n $
    \\$ \gcd(F_m, F_n) = F_{\gcd(m, n)} $
    \\$ F_m \mid F_n^2 \Leftrightarrow nF_n \mid m $
    \\$ F_n = \frac{\varphi^n - \varPsi^n}{\sqrt{5}}, \varphi = \frac{1 + \sqrt{5}}{2}, \varPsi = \frac{1 - \sqrt{5}}{2} $
    \\$ F_n = \lfloor \frac{\varphi^n}{\sqrt{5}} + \frac{1}{2} \rfloor, n \geq 0 $, $ n(F) = \lfloor \log_\varphi(F \cdot \sqrt{5} + \frac{1}{2}) \rfloor $

\subsection*{Stirling number (1st kind)}
    用$ {n \brack k} $表示Stirling number (1st kind),为将$ n $个元素分成$ k $个环的方案数
    \\$ {n + 1 \brack k} = n {n \brack k} + {n \brack k - 1}, k > 0 $
    \\$ {0 \brack 0} = 1, {n \brack 0} = {0 \brack n} = 0, n > 0 $
    \\$ {n \brack k} $为将$ n $个元素分成$ k $个环的方案数
    \\$ {x \brack x - n} = \sum\limits_{k = 0}^{n} \Eularian{n}{k} \binom{x + k}{2 n} $
    \\$ x^{\underline n} = \sum\limits_{k = 0}^{n} {n \brack k} (-1)^{n - k} x^k $
    \\$ x^{\overline n} = \sum\limits_{k = 0}^{n} {n \brack k} x^k $

\subsection*{Stirling number (2nd kind)}
    用$ {n \brace k} $表示Stirling number (2nd kind),为将$ n $个元素划分成$ k $个非空集合的方案数
    \\$ {n + 1 \brace k} = k {n \brace k} + {n \brace k - 1}, k > 0 $, $ {0 \brace 0} = 1, {n \brace 0} = {0 \brace n} = 0, n > 0 $
    \\$ {n \brace k} = \frac{1}{k!} \sum\limits_{j = 0}^{k} (-1) ^ {k - j} \binom{k}{j} j^n $, $ {x \brace x - n} = \sum\limits_{k = 0}^{n} \Eularian{n}{k} \binom{x + n - k - 1}{2 n} $

\subsection*{Catalan number}
    $ c_n $表示长度为$ 2n $的合法括号序的数量
    \\$ c_1 = 1 $, $ c_{n+1} = \sum\limits_{i=1}^{n} c_i \times c_{n + 1 - i} $, $ c_n = \frac{\binom{2n}{n}}{n + 1} $

\subsection*{Bell number}
    $ B_n $表示基数为$ n $的集合的划分方案数
    \\$ B_i = \begin{cases}
        1 & i = 0\\
        \sum\limits_{k = 0}^{i - 1} \binom{i - 1}{k} B_k & i > 0
    \end{cases} $
    \\$ B_n = \sum\limits_{k = 0}^{n} {n \brace k} $
    \\$ B_{p^m + n} \equiv m B_n + B_{n + 1} \pmod p $

\subsection*{五边形数定理}
    $ p(n) $表示将$ n $划分为若干个正整数之和的方案数
    \\$ p(n) = \sum\limits_{k \in \mathbb{N}^\ast} (-1)^{k - 1} p(n - \frac{k(3k - 1)}{2}) $

\subsection*{Bernoulli number}
    \noindent$ \sum\limits_{j = 0}^{m} \binom{m + 1}{j} B_j = 0, m > 0 $
    \\$ B_i = \begin{cases}
        1 & i = 0\\
        -\frac{\sum\limits_{j = 0}^{i - 1} \binom{i + 1}{j} B_j}{i + 1} & i > 0
    \end{cases} $
    \\$ \sum\limits_{k = 1}^{n} k ^ m = \frac{1}{m + 1} \sum\limits_{k = 0}^{m} \binom{m + 1}{k} B_k n ^ {m + 1 - k} $
    \\生成函数$ \sum\limits_{i = 0}^{\infty} B_i \frac{x^i}{i!} = \frac{x}{e^x - 1} = \frac{1}{\sum\limits_{i = 0}^{\infty} \frac{x^i}{(i + 1)!}} $

\subsection*{Stirling permutation}
    $ 1, 1, 2, 2 \dots , n, n $的排列中,对于每个$ i $,都有两个$ i $之间的数大于$ i $
    \\排列方案数为$ (2n - 1)!! $

\subsection*{Eulerian number}
    $ \eularian{n}{k} $表示$ 1 $到$ n $的排列中,恰有$ k $个数比前一个大的方案数
    \\$ \eularian{n}{0} = \eularian{n}{n - 1} = 1 $, $ \eularian{0}{m} = [m = 0] $
    \\$ \eularian{n}{m} = \eularian{n}{n - 1 - m} $
    \\$ \eularian{n}{m} = (m + 1) \eularian{n - 1}{m} + (n - m) \eularian{n - 1}{m - 1} $
    \\$ \eularian{n}{m} = \sum\limits_{k = 0}^{m} (-1)^k \binom{n + 1}{k} (m + 1 - k)^n $

\subsection*{Eulerian number (2nd kind)}
    $ \Eularian{n}{k} $表示Stirling permutation中,恰有$ k $个数比前一个大的方案数
    \\$ \Eularian{n}{m} = (2n - m - 1) \Eularian{n - 1}{m - 1} + (m + 1) \Eularian{n - 1}{m} $
    \\$ \Eularian{n}{0} = 1 $, $ \Eularian{0}{m} = [m = 0] $

\subsection*{二项式反演}
    $ f(n) = \sum\limits_{i = 0}^{n} \binom{n}{i} g(i) \Leftrightarrow g(n) = \sum\limits_{i = 0}^{n} (-1)^{n - i} \binom{n}{i} f(i) $
    \\$ f(n) = \sum\limits_{i = 0}^{n} (-1)^i \binom{n}{i} g(i) \Leftrightarrow g(n) = \sum\limits_{i = 0}^{n} (-1)^i \binom{n}{i} f(i) $

\subsection*{Stirling反演}
    $ f(n) = \sum\limits_{i = 0}^{n} { n \brace i} g(i) \Leftrightarrow g(n) = \sum\limits_{i = 0}^{n} (-1)^{n - i} {n \brack i} f(i) $

\subsection*{Möbius function}
    $ \mu(n) = \begin{cases}
        1 & n \text{ square-free, even number of prime factors}\\
        -1 & n \text{ square-free, odd number of prime factors}\\
        0 & n \text{ has a squared prime factor}
    \end{cases} $
    \\$ \sum\limits_{d \mid n} \mu(d) = \begin{cases}
        1 & n = 1\\
        0 & n > 1
    \end{cases} $
    \\$ f(n) = \sum\limits_{d \mid n} g(d) \Leftrightarrow g(n) = \sum\limits_{d \mid n} \mu(d) f(\frac{n}{d}) $
    \\$ f(n) = \sum\limits_{n \mid d} g(d) \Leftrightarrow g(n) = \sum\limits_{n \mid d} \mu(\frac{d}{n}) f(d) $

\subsection*{Lagrange polynomial}
    给定次数为$ n $的多项式函数$ L(x) $上的$ n + 1 $个点$ (x_0, y_0), (x_1, y_1), \dots, (x_n, y_n) $
    \\则$ L(x) = \sum\limits_{j = 0}^{n} y_j \prod\limits_{0 \leq m \leq n, m \ne j} \frac{x - x_m}{x_j - x_m} $

\subsection*{Burnside lemma}
    Let $ G $ be a finite group that acts on a set $ X $. For each $ g $ in $ G $ let $ X^g $ denote the set of elements in $ X $ that are fixed by $ g $ (also said to be left invariant by $ g $), i.e. $ X^g = \lbrace x \in X \mid g.x = x \rbrace $. Burnside's lemma asserts the following formula for the number of orbits, denoted $ \left| X / G \right| $:
    \\$ \left| X / G \right| = \frac{1}{\left| G \right|} \sum\limits_{g \in G}^{} \left| X^g \right| $

\subsection*{Pólya theorem}
    设$ \overline{G} $是$ n $个对象的置换群,用$ m $种颜色对$ n $个对象染色,则不同染色方案为:
    \\$ L = \frac{1}{\left| \overline{G} \right|} (m^{c(\overline{P_1})} + m^{c(\overline{P_2})} + \dots + m^{c(\overline{P_g})}) $
    \\其中$ \overline{G} = \lbrace \overline{P_1}, \overline{P_2}, \dots, \overline{P_g} \rbrace $,$ c(\overline{P_k}) $为$ \overline{P_k} $的循环节数
